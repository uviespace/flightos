\usepackage[xindy, nopostdot, numberedsection, style=super, section, toc, acronyms, nogroupskip]{glossaries}
\usepackage{xparse}

\setlength{\glsdescwidth}{0.8\textwidth}
\renewcommand{\glsnamefont}[1]{\textbf{#1}}

% label, acronym, name, description
\DeclareDocumentCommand{\newdualentry}{ m m m m } {
	\newglossaryentry{gls-#1}{
	name={#3 (\gls{#1})},
	description={#4}, nonumberlist
	}
	\newglossaryentry{#1}{
		type=\acronymtype,
		name={#2},
		first={#3 (#2)},
		firstplural={#3s (#2s)},
		see={[Glossary:]{\gls{gls-#1}}},
		description=\glslink{gls-#1}{#3},
		nonumberlist
	}%
}
%%% \gls{ABBREV} to point at Acronym/Abbreviation index
%%% \gls{gls-ABBREV} to point to glossary


\newdualentry{ADC}%
  {ADC}%
  {Analog to Digital Converter}%
  {An Analog to Digital Converter is a system that converts an analog signal
   into a quantized digital signal. Its counterpart is the \gls{DAC}.}%

\newdualentry{API}%
  {API}%
  {Application Programming Interface}%
  {The Application Programming Interface defines how a developer can write
   a program that requests services from an operating system or application.
   \glspl{API} are implemented by function calls composed of verbs and nouns,
   i.e. a function to execute on an object.}%

\newdualentry{BSP}%
  {BSP}%
  {Board Support Package}%
  {A Board Support Package is the implementation of a specific interface defined
   by the abstract layer of an operating system that enables the latter to run
   on the particular hardware platform.}%

\newdualentry{CPU}%
  {CPU}%
  {Central Processing Unit}%
  {The Central Processing Unit is the electronic circuitry that interprets
  instructions of a computer program and performs control logic, arithmetic,
  and input/output operations specified by the instructions. It maintains
  high-level control of peripheral components, such as memory and other devices.}%

\newdualentry{DAC}%
  {DAC}%
  {Digital to Analog Converter}%
  {A Digital to Analog Converter is a system that converts a quantized digital
   signal into an analog signal. Its counterpart is the \gls{ADC}.}%

\newdualentry{DMA}%
  {DMA}%
  {Direct Memory Access}%
  {Direct Memory Access is a feature of a computer system that allows hardware
   subsystems to access main system \gls{gls-RAM} directly, thereby bypassing
   the \gls{gls-CPU}.}%

\newacronym{DRD}{DRD}{Document Requirements Definition}

\newdualentry{DSP}%
  {DSP}%
  {Digital Signal Processor}%
  {A Digital Signal Processor is a specialised processor with its architecture %
   targeting the operational needs of digital signal processing.}%

\newdualentry{ELF}%
  {ELF}%
  {Executable and Linkable Format}%
  {The Executable and Linkable Format is a common standard file format for
   executables, object code, shared libraries, and core dumps.}%

\newacronym{FDIR}{FDIR}{Fault Detection, Isolation and Recovery}

\newdualentry{FIFO}%
  {FIFO}%
  {First In - First Out}%
  {In FIFO processing, the "head" element of a queue is processed first.
   Once complete, the element is removed and the next element in line becomes
   the new queue head.}%

\newdualentry{FPU}%
  {FPU}%
  {Floating Point Unit}%
  {A co-processor unit that specialises in floating-point calculations.}%

\newdualentry{ILP}%
  {ILP}%
  {Instruction Level Parallelism}%
  {Instruction-level parallelism (ILP) is a measure of how many instructions in
   a computer program can be executed simultaneously by the \gls{CPU}.}%

\newdualentry{ISR}%
  {ISR}%
  {Interrupt Service Routine}%
  {An Interrupt Service Routine is a function that handles the actions needed
   to service an interrupt.}%

\newacronym{FSW}{FSW}{Flight Software}

\newdualentry{GCC}%
  {GCC}%
  {GNU Compiler Collection}%
  {The GNU Compiler Collection is a compiler system produced by the
   GNU project. It is part of the GNU toolchain collection of programming
   tools.}%

\newglossaryentry{GNU}{
  name={GNU},
  description={GNU is a recursive acronym that stands for "GNU's Not Unix!".
	      The GNU project is a free software collaboration project announced
	      in 1987. Users are free to run GNU-licensed software, share, copy,
	      distribute, study and modify it. GNU software guarantees these
	      freedom-rights legally via its license and is thefore free
      	      software.},
  nonumberlist
}

\newglossaryentry{LEON2}{
  name={LEON2},
  description={The LEON2 is a synthesisable VHDL model of a 32-bit processor
	       compliant with the SPARC V8 architecture. It is highly
	       configurable and particularly suitable for \gls{SoC} designs.
	       Its source code is available under the GNU LGPL license},
  nonumberlist
  }

\newglossaryentry{LEON3}{
  name={LEON3},
  description={The LEON3 is an updated version of the \gls{LEON2}, changes
	       include \gls{gls-SMP} support and a deeper instruction pipeline},
  nonumberlist
  }

\newglossaryentry{LEON3-FT}{
  name={LEON3-FT},
  description={The LEON3-FT is a fault-tolerant version of the \gls{LEON3}.
  	       Changes to the base version include autonomous error handling,
       	       cache locking and different cache replacement strategies.},
  nonumberlist
  }

\newdualentry{MMU}%
  {MMU}%
  {Memory Management Unit}%
  {A Memory Management Unit performs address space translation between physical
   and virtual memory pages and protects unprivileged access to certain memory
   regions.}%

\newdualentry{MPPB}%
  {MPPB}%
  {Massively Parallel Processor Breadboarding system}%
  {The Massively Parallel Processor Breadboarding system is a proof-of-concept %
   design for a space-hardened, fault-tolerant multi-DSP system with various %
   subsystems to build a powerful digital signal processing system with a high %
   data throughput. Its distinguishing features are the \gls{gls-NoC} and the
   \gls{Xentium} \glspl{DSP} controlled by a \gls{LEON2} processor.
   It was developed under ESA contract 21986 by Recore Systems B.V.}%

\newacronym{NCR}{NCR}{Non-Conformance Reports}

\newdualentry{NGAPP}%
  {NGAPP}%
  {Next Generation Astronomy Processing Platform}%
  {Next Generation Astronomy Processing Platform was an evaluation of the
   \gls{MPPB} performed in a joint effort of RUAG Space Austria and the
   Department of Astrophysics of the University of Vienna.
   The project was funded under ESA contract 40000107815/13/NL/EL/f.}%

\newdualentry{NoC}%
  {NoC}%
  {Network On Chip}%
  {A Network On Chip is a communication system on an integrated circuit that
   applies (packet based) networking to on-chip communication. It offers
   improvements over more conventional bus interconnects and is more scalable
   and power efficient in complex \gls{gls-SoC} desgins.}%

\newacronym{NRB}{NRB}{Nonconformance Review Board}

\newacronym{PA}{PA}{Product Assurance}

\newacronym{PAM}{PAM}{Product Assurance Manager}

\newdualentry{POSIX}
  {POSIX}
  {Portable Operating System Interface}
  {The Portable Operating System Interface is a family of standards specified
   by the IEEE Computer Society for maintaining compatibility between
   operating systems.}%

\newdualentry{PUS}
  {PUS}
  {Packet Utilisation Standard}
  {The Packet Utilisation Standard addresses the end-to-end transport of telemetry
   and telecommand data between user applications on the ground and applications
   onboard a satellite. See also ECSS-E-70-41A.}%

\newdualentry{RAM}%
  {RAM}%
  {Random-Access Memory}%
  {Random-Access Memory is a type of memory where each memory cell may be
   accessed directly via their memory addresses.}%

\newacronym{RID}{RID}{Review Item Discrepancy}

\newdualentry{RISC}%
  {RISC}%
  {Reduced Instruction Set Computing}%
  {RISC is a \gls{CPU} design strategy that intends to improve performance by
   combining a simplified instruction set with a microprocessor architecture
   that is capable of executing an instruction in a smaller number of clock
   cycles.}%

\newdualentry{RMAP}%
  {RMAP}%
  {Remote Memory Access Protocol}%
  {The Remote Memory Access Protocol is a form of \gls{SpaceWire} communication
   that transparently communicates writes to memory mapped regions between
   different hardware devices.}%


\newdualentry{RR}%
  {RR}%
  {Round Robin}%
  {Round Robin is a scheduling algorithm where time slices are assigned in equal
   poritions and in circular order. In the context of threads, priorities are
   usually only used to control re-scheduling order when a mutex is accessed by
   a thread.}%

\newacronym{RSA}{RSA}{RUAG Space Austria}

\newacronym{SDD}{SDD}{Software Design Document}

\newacronym{SDP}{SDP}{Software Development Plan}

\newdualentry{SMP}%
  {SMP}%
  {Symmetric Multiprocessing}%
  {Symmetric Multiprocessing denotes computer architectures, where two or more
   identical processors are connected to the same periphery and are controlled
   by the same operating system instance.}%

\newacronym{SPAMR}{SPAMR}{Software Product Assurance Milestone Report}

\newacronym{SPR}{SPR}{Software Problems Reports}

\newacronym{SSS}{SSS}{System Software Specification}

\newacronym{SRS}{SRS}{Software Requirement Specification}

\newdualentry{SoC}%
  {SoC}%
  {System On Chip}%
  {A System On Chip is an integrated circuit that combines all components of a %
   computer or other electronic system into a single chip.}%

\newglossaryentry{SpaceWire}{
  name={SpaceWire},
  description={SpaceWire is a spacecraft communication network based in part
               on the IEEE 1355 standard of communications.},
  nonumberlist
  }

\newglossaryentry{SPARC}{
  name={SPARC},
  description={SPARC ("scalable processor architecture") is a \gls{gls-RISC}
  	       instruction set architecture developed by Sun Microsystems in
	       the 1980s. The distinct feature of SPARC processors is the high
	       number of \gls{gls-CPU} registers that are accessed similarly to
	       stack variables via ``sliding windows''.},
  nonumberlist
  }


\newacronym{SQ}{SQ}{Software Quality}

\newacronym{SQAM}{SQAM}{Software Quality Assurance Manager}

\newdualentry{SSDP} % label
  {SSDP}            % abbreviation
  {Scalable Sensor Data Processor}  % long form
  {The Scalable Sensor Data Processor (SSDP) is a next generation on-board %
   data processing mixed-signal ASIC, envisaged to be used in future scientific %
   payloads requiring high-performance on-board processing capabilities. %
   It is built opon a heterogeneous multicore architecture, combining two %
   \gls{Xentium} \gls{DSP} cores with a general-purpose \gls{LEON3-FT} control %
   processor in a \gls{gls-NoC}.} % description

\newacronym{SQA}{SQA}{Software Quality Assurance}

\newdualentry{TCM}%
  {TCM}%
  {Tightly-Coupled Memory}%
  {Tightly-Coupled Memory is the local data memory that is directly accessible %
   by a Xentium's load/store unit. It can be viewed as a completely %
   program-controlled data cache.}%


\newacronym{TP}{TP}{Test Plan}
\newacronym{TS}{TS}{Test Specification}

\newacronym{UVIE}{UVIE}{University of Vienna}

\newdualentry{VLIW}%
  {VLIW}%
  {Very Long Instruction Word}%
  {Very Long Instruction Word is a processor architecture design concept that
   exploits \gls{gls-ILP}. This approach allows higher performance at a smaller
   silicone footprint compared to serialised instruction processors, as no
   instruction re-ordering logic to exploit superscalar capabilities of the
   processor must be integrated on the chip, but requires either code to be
   tuned manually or a very sophisticated compiler to exploit the full potential
   of the processor.}%

\newglossaryentry{Xentium}{
  name={Xentium},
  description={The Xentium is a high performance \gls{gls-VLIW} \gls{DSP} core.
               It operates 10 parallel execution slots supporting 32/40 bit
	       scalar and two 16-bit element vector operations.},
  nonumberlist
  }


% do not remove
\glsresetall
\makeglossaries
